\documentclass[10pt]{beamer}
\setbeamerfont{structure}{family=\rmfamily} 
\usepackage{amsthm}
\usepackage{graphicx}
\usepackage{graphics}
\usepackage{hyperref}
\beamertemplatenavigationsymbolsempty
\setbeamertemplate{blocks}[rounded][shadow=true]
\setbeamertemplate{bibliography item}[text]
\setbeamertemplate{caption}[numbered]
\usetheme{default} 
\usecolortheme{seahorse}
\mode<presentation>
{
   \setbeamercovered{transparent}
   \setbeamertemplate{items}[ball]
   \setbeamertemplate{theorems}[numbered]
   \setbeamertemplate{footline}[frame number]

}

\begin{document}
\title {\bfseries{\sc Emulator for Hazard Analysis:\\ A Usage tutorial}}
\author {Prof. Abani Patra}
\institute{University	at	Buffalo }
\date{\small 15-17 March , 2016} 
%--------------------------------------------------------------------------%
\begin{frame}
\titlepage
\end{frame}
%---------------------------------------------------------------------------%
\section*{OUTLINE}
\begin{frame}
\frametitle{OUTLINE}  
\tableofcontents
\end{frame}
%---------------------------------------------------------------------------

\section{Background}
\begin{frame}
\frametitle{Background}
Direct Computation of Elevation height and Particle flux may be computationally very expensive.\\
\vspace{1mm}
For handling the complexity, a simple Emulator could be constructed. \\
Some features of the presented model are:\\
\begin{itemize} 
    \item  Assumes the underlying distribution as a Gaussian Stochastic process
    \item Capable of studying the correlation between the input parameters
    \item The parameters for the fitted model can be computed through MLE as well as REML
    \item For providing an intuition regarding accuracy, the display panel shows the location of the LHS generated points on the mean surface for the gaussian distribution.
    \item For validating the model, the emulator also provides a visual description of the residuals of the sample points along with the standard errors for sample as well as LHS points

\end{itemize}
\end{frame}
%---------------------------------------------------------------------------%.

\section{MPErK}
\begin{frame}
\frametitle{MPErK (a MATLAB program for Parametric Empirical Kriging)}
Views computer experiments as gaussian stochastic process.\\
\begin{align*}
     Y(x)= \beta^Tf(x) + z(x)   
\end{align*}

\vspace{3mm}

 
f(x): px1 vector of regression functions\\
\(\beta\): Unknown vector of regression parameters\\
z(x): zero mean covariance 

\vspace{2mm}

For any inputs x1 and x2
\begin{align*}
     Cov(z(x1),z(x2))= \sigma^{2}R((x1-x2)/{\epsilon} )   
\end{align*}

\(\epsilon\) can be computed through MLE or REML\\
R could represent Gaussian, PowerExponential or Cubic correlation functions
\end{frame}

%---------------------------------------------------------------------------%
\section{Emulator}
\begin{frame}
\frametitle{Emulator Usage}
Using the constructed emulator is fairly simple and involve following steps:\\

{\bf Input}:
\begin{itemize}
    \item Input the data as Follows
    \begin{itemize}
        \item The data should be passed in a .csv format
        \item There should be four columns in the csv file in the order: Water Fraction, Temperature, followed by corresponding Eruption height and Particle flux
        \item For passing the data, just click the 'UPLOAD' button in the emulator and select your .csv file (in All Files option)
   \end{itemize}
    \item Select the type of Correlation functions which seem to be most suitable for the data
    \item Selection the optimization criterion for determining the correlation parameter
    \item Cross validation could be carried out based on the options mentioned in analysis part
\end{itemize}

\end{frame}
%-------------------------------------------------------------------------------%
\begin{frame}
In order to analyze the performance of the constructed model, following visualization could be used \\
\vspace{4mm}
{\bf Analysis}:
\begin{itemize}
    \item Standard Error for the LHS generated dataset
    \item Standard Error for the Input sample
    \item Residuals for the Input sample
\end{itemize}
\vspace{4mm}
Based on the visualizations for the gaussian mean surface as well as plots for the residuals and standard error, the decisions regarding the underlying correlation structure could be made\\
\vspace{2mm}
The values for Variance and Mean residuals also help in determining the most suitable correlation function 
\end{frame}

%---------------------------------------------------------------------------%
\section{Predictor}
\begin{frame}
\frametitle{Prob(Eruption height \textless `Height')}

\begin{itemize}
    \item Enter the `Height' to be inspected (in meters)
    \item In the background, the model generated samples from a Uniform distribution with the same parameters as done by Puffin
    \item When the 'Calculate' button is clicked, these random samples are evaluated for eruption height through the coefficients of the Gaussian fitted model
    \item Based on the proportion of the samples, which produce eruption height less than the entered height, the probability value is computed
\end{itemize}

\end{frame}




% \begin{frame}{References}
%   \begin{thebibliography}{99}
%   \bibitem{one}
% Anis Das Sharma, Alpa Jain, Kong Yu, " Dynamic Relationship and Event Discvery".
% \bibitem{two}
% Nguyen Bach and Sameer Badaskar, Presentation on "Survey on Relation Extraction".
% \bibitem{three}
% Sunita Sarawagi, "Surv"

% \end{thebibliography}
% \end{frame}
% %---------------------------------------------------------------------------%

\begin{frame}
\Large
\begin{center}
 \sc {Thank You \ldots} 
\end{center}
\end{frame}
%---------------------------------------------------------------------------%
\end{document}
