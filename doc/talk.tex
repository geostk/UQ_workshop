\documentclass[10pt]{beamer}
\setbeamerfont{structure}{family=\rmfamily} 
\usepackage{amsthm}
\usepackage{graphicx}
\usepackage{graphics}
\usepackage{hyperref}
\beamertemplatenavigationsymbolsempty
\setbeamertemplate{blocks}[rounded][shadow=true]
\setbeamertemplate{bibliography item}[text]
\setbeamertemplate{caption}[numbered]
\usetheme{default} 
\usecolortheme{seahorse}
\mode<presentation>
{
   \setbeamercovered{transparent}
   \setbeamertemplate{items}[ball]
   \setbeamertemplate{theorems}[numbered]
   \setbeamertemplate{footline}[frame number]

}

\begin{document}
\title {\bfseries{\sc Uncertainty in Geo-science:\\ A Workshop on Hazard Analysis \\ {\small Hands on Module} }}
\author { H. Aghakhani, Z. Cao, P. Shekhar}
\institute{University	at	Buffalo }
\date{\small 15-17 March , 2016} 
%--------------------------------------------------------------------------%
\begin{frame}
\titlepage
\end{frame}
%---------------------------------------------------------------------------%
\section*{OUTLINE}
\begin{frame}
\frametitle{OUTLINE}  
\tableofcontents
\end{frame}
%---------------------------------------------------------------------------

\section{Problem Definition}
\begin{frame}
\frametitle{Problem Statement}
Study of  uncertainty in input parameters of complex computer models: 
\begin{itemize}
    \item  {\bf Intrusive}: change the original governing equation.
    \begin{itemize}
        \item Polynomial Chaos / Stochastic Galerkin.
        \item Perturbation methods
        \item ...
    \end{itemize}
    \item  {\bf Non-intrusive}: \textit{ do not} change the original governing equation.
    \begin{itemize}
        \item Monte Carlo \& LHS
        \item Important sampling Methods
        \item Non-intrusive spectral projection (NISP)
        \item ...
    \end{itemize}
\end{itemize}
\end{frame}
%---------------------------------------------------------------------------%
\section{LHS}
\begin{frame}
\frametitle{Latin Hypercube Sampling (LHS)}
Generating a sample set of uncertain values from equally probable intervals of the probability density function.\\

\begin{enumerate}
    \item Select the pdf
    \item Select the number of samples
    \item Divide the pdf function into equal probability intervals.
    \item Generate random samples on each interval.
    \item {\it Starting point for other methods -- Gaussian Processes, BLM etc.}
\end{enumerate}
{\bf Advantage}:
\begin{itemize}
    \item Fewer samples require for convergence compared to MC
    \item No need to change the original solver (like any other non-intrusive method)
\end{itemize}
%---------------------------------------------------------------------------%
\end{frame}
%---------------------------------------------------------------------------%
\section{Instructions for LHS script}
\begin{frame}
\frametitle{Instructions for LHS script}
To run LHS\_UQ you just need to do the following:
 \begin{itemize}
 
 \item Open a python interpreter by typing python on your terminal.
 
 \item To load the LHS\_UQ function type: from LHS import LHS\_UQ 
 
 \item To run the function you need to provide the following arguments by typing:\newline LHS\_UQ(num\_samples , min\_wat\_cont , range\_ wat\_cont , min\_temp ,range\_temp):
   \begin{enumerate}
 \item number of samples
 \item minimum of water content
 \item range of water content
 \item minimum of temperature
 \item range of temperature
 \end{enumerate}

 \end{itemize}

\end{frame}
%-------------------------------------------------------------

--------------------
\section{PCQ}

\begin{frame}
\frametitle{Polynomial Chaos Quadrature (PCQ)}
The basic idea comes from projection theory that each function can be written as an expansion of a series of orthogonal functions:\\
\begin{equation}
     h(\eta)= \sum_{i=1}^\infty a_i \Psi_i (\eta) \label{eq:pcq-expansion}  
\end{equation}
So any uncertain parameter in the model can be expressed as above. \\
The inner product of $h(\eta)$ and $j$th basis function $\psi_j$ is $<h(\eta), \Psi_j (\eta) > \equiv \int_{-\infty}^{\infty} h(\eta) \Psi_j (\eta) \rho(\eta) d \eta $ $\rho(x)$ is a weight function
\begin{equation}
     <h(\eta), \Psi_j (\eta) >= <\sum_i a_i \Psi_i (\eta), \Psi_j (\eta)>\label{eq:pcq-inner-product}
\end{equation}
Due to orthogonality of basis:
\begin{equation}
     <h(\eta), \Psi_j (\eta) >= a_i<\Psi_i (\eta), \Psi_j (\eta)> \label{eq:pcq-inner-product-orthogonal}
\end{equation}
\begin{equation}
     a_i= \frac{<h(\eta), \Psi_j (\eta) >}{<\Psi_i (\eta), \Psi_i (\eta)>}\label{eq:pcq-coef}
\end{equation}
\end{frame}
%-------------------------------------------------------------
%---------------------------------------------------------------------------%
\begin{frame}
\frametitle{Polynomial Chaos Quadrature (PCQ)}
The integration is then approximated using numerical quadrature.\\
By definition, inner product with respect to certain distribution is:
\begin{equation}
\begin{split}
      <h(\eta), \Psi_j (\eta) > &= \int_{-\infty}^{\infty} h(\eta) \Psi_j (\eta) \rho(\eta) d \eta \\
      &= \sum_i h(\eta_i) \Psi_j (\eta_i) w_i 
\end{split} \label{eq:pcq-gaussian-h} 
\end{equation}
\begin{equation}
      <\Psi_j (\eta), \Psi_j (\eta) > = \sum_k \Psi_j (\eta_k) \Psi_j (\eta_k) w_k \label{eq:pcq-gaussian-psi} 
\end{equation}
\end{frame}
%-------------------------------------------------------------
%---------------------------------------------------------------------------%
\begin{frame}
\frametitle{Polynomial Chaos Quadrature (PCQ)}
For some cases the orthogonality of basis can not be guaranteed, then the following system of equations need to be solved for coefficient evaluation.
%\begin{equation}
% h(\eta_k)= \sum_i a_i \Psi_i (\eta_k) 
% \label{eq:system-eq}
%\end{equation}

\begin{equation}
    <\sum_i a_i \Psi_i (\eta), \Psi_j (\eta)>   = <h(\eta), \Psi_j (\eta) >
\end{equation}
\end{frame}
%-------------------------------------------------------------
%---------------------------------------------------------------------------%
\section{Instructions for PCQ script}
\begin{frame}
\frametitle{Instructions for PCQ script (quick start)}
To run PCQ\_UQ you just nead to do the following:
 \begin{itemize}
 
 \item Open a python interpreter by typing python on your terminal.
 
 \item To load the PCQ\_UQ function type: from PCQ import PCQ\_UQ 
 
 \item To run the function you need to provide the following arguments by typing:\newline PCQ\_UQ (num\_samples, mean\_rv, range\_rv, v\_dis, v\_name, rv\_value, alf, beta):
  \begin{enumerate}
 \item number of samples
 \item mean of random variables
 \item range of random variables
 \item name of distribution for each random variables
 \item name of two random parameters
 \item default value of two random parameters (222 for speed, 333 for radius, 0.33 for water fraction, 1111 for temperature)
 \item alf and beta are parameters for "Beta" distribution
 \end{enumerate}

 \end{itemize}

\end{frame}
%-----------------------------------------
%---------------------------------------------------------------------------%
\begin{frame}
\frametitle{Instructions for PCQ script (quick start)}
For example:\\
  PCQ\_UQ ([2,2], [200, 200], [100, 100], ["Uniform", "Uniform"], ['radius', 'speed'], [222, 333], 0.5, 0.5) \\
 For this example, the input parameters are:
 \begin{enumerate}
 \item 2 Gaussian quadrature points for each random input parameter.
 \item Mean for each RV is 200
 \item Range of each RV is 100
 \item Assume "Uniform" distribution for both RVs
 \item Select radius and speed as input random parameters
 \item The default value
 \item $\alpha$ and $\beta$ is useless in this case
 \end{enumerate}

\end{frame}
%-------------------------------------------------------------
%---------------------------------------------------------------------------%
\begin{frame}
\frametitle{Instructions for PCQ script (more details)}
The basic procedure of running the PCQ script:
 \begin{itemize}
 \item Generate sample points and quadrature weight
 \item Run simulation with sampled parameters
 \item Parse output file and extract desired properties
 \item Prepare for coefficient computing
 \item compute coefficient by PCQ (or by solving system of equations)
 \item Post process
 \end{itemize}
\end{frame}
%-------------------------------------------------------------
%---------------------------------------------------------------------------%
\begin{frame}
\frametitle{Instructions for PCQ script(more details)}
Before starting, you have to determine:
 \begin{itemize}
 \item {\bf Random parameters}\\
 The recommended input parameters are: eruption speed, vent radius, mass fraction of water in erupted material, temperaure of erupted material.
 \item {\bf Underlying distribution} \\
  For each random variable (given by dis\_i in the script)
 \item {\bf Mean and range} \\
  For each random parameters. (given by mean\_xxx and range\_xxx in the script)
 \item {\bf Number of smaple points} \\
 For each random variable
 \end{itemize}
\end{frame}
%-------------------------------------------------------------
%---------------------------------------------------------------------------%
\begin{frame}
\frametitle{Instructions for PCQ script(more details)}
\begin{itemize}  
 \item {\bf Generate sample points and quadrature weight}:\\
   \begin{itemize}
    \item The Gaussian Quadrature points generator should be called accordingly based on distribution that you selected:
    Hermite for Gaussian, Legendre for Uniform, Leguerre for Gamma, and Jacobi for Beta (modify smplingx=...)
    \item Transfer and scale sample points
    \end{itemize}
 \item {\bf Run simulations with sampled parameters}:\\
 This will be done automatically. 
 \item {\bf Parse output file and extract desired properties}:\\
 The output is eruption mass flux and plume height. These two properties will be extracted with parse script automatically.
 \item {\bf Prepare for coefficient computing}:\\
 Basically, just re-organize array into matrix of desired dimension.
\end{itemize}
\end{frame}
%-------------------------------------------------------------
%---------------------------------------------------------------------------
 \begin{frame}
 \frametitle{Instructions for PCQ script(more details)}
 \begin{itemize}
 \item {\bf compute coefficient by PCQ (or by solving system of equations)}:\\
 When the orthogonality of the basis is guaranteed, use PCQ or more general method (sovling system of equations). Otherwise, only the more general method could be used. You can switch from these two method by commentting off/on.
 \item {\bf Post process}:\\ 
  \begin{itemize}
   \item Plot plume height (or mass flux) as a function of random parameters. num\_plot\_xxx gives number of re-sample points for plot.
   \item Plot histogram of plume height (or mass flux) for specific distribution of input parameters.  
 \end{itemize}
 \item {\bf Usage of subroutines is given very explicitly in the script} 
\end{itemize}
\end{frame}
%-------------------------------------------------------------
%---------------------------------------------------------------------------%


% \begin{frame}{References}
%   \begin{thebibliography}{99}
%   \bibitem{one}
% Anis Das Sharma, Alpa Jain, Kong Yu, " Dynamic Relationship and Event Discvery".
% \bibitem{two}
% Nguyen Bach and Sameer Badaskar, Presentation on "Survey on Relation Extraction".
% \bibitem{three}
% Sunita Sarawagi, "Surv"

% \end{thebibliography}
% \end{frame}
% %---------------------------------------------------------------------------%

\begin{frame}
\Large
\begin{center}
 \sc {Thank You \ldots} 
\end{center}
\end{frame}
%---------------------------------------------------------------------------%
\end{document}
