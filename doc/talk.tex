\documentclass[10pt]{beamer}
\setbeamerfont{structure}{family=\rmfamily} 
\usepackage{amsthm}
\usepackage{graphicx}
\usepackage{graphics}
\usepackage{hyperref}
\beamertemplatenavigationsymbolsempty
\setbeamertemplate{blocks}[rounded][shadow=true]
\setbeamertemplate{bibliography item}[text]
\setbeamertemplate{caption}[numbered]
\usetheme{default} 
\usecolortheme{seahorse}
\mode<presentation>
{
   \setbeamercovered{transparent}
   \setbeamertemplate{items}[ball]
   \setbeamertemplate{theorems}[numbered]
   \setbeamertemplate{footline}[frame number]

}

\begin{document}
\title {\bfseries{\sc Uncertainty in Geo-science:\\ A Workshop on Hazard Analysis}}
\author {Prof. Abani Patra}
\institute{University	at	Buffalo }
\date{\small 15-17 March , 2016} 
%--------------------------------------------------------------------------%
\begin{frame}
\titlepage
\end{frame}
%---------------------------------------------------------------------------%
\section*{OUTLINE}
\begin{frame}
\frametitle{OUTLINE}  
\tableofcontents
\end{frame}
%---------------------------------------------------------------------------

\section{Problem Definition}
\begin{frame}
\frametitle{Problem Statement}
Study of physical phenomena involved with uncertainty in input parameters: 
\begin{itemize}
    \item  {\bf Intrusive}: change the original governing equation.
    \begin{itemize}
        \item Polynomial Chaos Stochastic Galerkin.
        \item Perturbation methods
        \item ...
    \end{itemize}
    \item  {\bf Non-intrusive}: \textit{ do not} change the original governing equation.
    \begin{itemize}
        \item Monte Carlo \& LHS
        \item Important sampling Methods
        \item Non-intrusive spectral projection (NISP)
        \item ...
    \end{itemize}
\end{itemize}
\end{frame}
%---------------------------------------------------------------------------%
\section{LHS}
\begin{frame}
\frametitle{Latin Hypercube Sampling (LHS)}
Generating a sample set of uncertain values from equally probable intervals of the probability density function.\\

\begin{enumerate}
    \item Select the pdf
    \item Select the number of samples
    \item Divide the pdf function into equal probability intervals.
    \item Generate random samples on each interval.
\end{enumerate}
{\bf Advantage}:
\begin{itemize}
    \item Much less number of samples require for convergence compared to MC
    \item No need to change the original solver (like any other non-intrusive method)
\end{itemize}


\end{frame}
%-------------------------------------------------------------------------------%
\section{PCQ}

\begin{frame}
\frametitle{Polynomial Chaos Quadrature (PCQ)}
The basic idea comes from projection theory that each function can be written as an expansion of a series of orthogonal function:\\
\begin{equation}
     h(\eta)= \sum_i a_i \Psi_i (\eta) \label{eq:pcq-expansion}  
\end{equation}
So any uncertain parameter in the model can be expressed as above. \\
The inner product of $h(\eta)$ and $j$th basis fucntion is
\begin{equation}
     <h(\eta), \Psi_j (\eta) >= <\sum_i a_i \Psi_i (\eta), \Psi_j (\eta)>\label{eq:pcq-inner-product}
\end{equation}
Due to linear property of the inner product and orthogonality of basis:
\begin{equation}
     <h(\eta), \Psi_j (\eta) >= a_i<\Psi_i (\eta), \Psi_j (\eta)> \label{eq:pcq-inner-product-orthogonal}
\end{equation}
\begin{equation}
     a_i= \dfrac{<\Psi_i (\eta), \Psi_j (\eta)>}{<h(\eta), \Psi_j (\eta) >} \label{eq:pcq-coef}
\end{equation}
\end{frame}
%-------------------------------------------------------------
%---------------------------------------------------------------------------%
\begin{frame}
\frametitle{Polynomial Chaos Quadrature (PCQ)}
The intergration is then approximated using the concept of the quadrature points.\\
By definition, inner product with respect to certain distribution is:
\begin{equation}
\begin{split}
      <h(\eta), \Psi_j (\eta) > &= \int_{-\infty}^{\infty} h(\eta) \Psi_j (\eta) p(\eta) d \eta \\
      &= \sum_i h(\eta_i) \Psi_j (\eta_i) w_i 
\end{split} \label{eq:pcq-gaussian-h} 
\end{equation}
\begin{equation}
      <\Psi_j (\eta), \Psi_j (\eta) > = \sum_k \Psi_j (\eta_k) \Psi_j (\eta_k) w_k \label{eq:pcq-gaussian-psi} 
\end{equation}
\end{frame}
%-------------------------------------------------------------
%---------------------------------------------------------------------------%
\section{Instructions for LHS script}
\begin{frame}
\frametitle{Instructions for LHS script}
To run LHS\_UQ you just nead to do the following:
 \begin{itemize}
 
 \item Open a python interpreter by typing python on your terminal.
 
 \item To load the LHS\_UQ function type: from LHS import LHS\_UQ 
 
 \item To run the function you need to provide the following arguments by typing:\newline LHS\_UQ(num\_samples , min\_wat\_cont , range\_ wat\_cont , min\_temp ,range\_temp):
   \begin{enumerate}
 \item number of samples
 \item minimum of water content
 \item range of water content
 \item minimum of temperature
 \item range of temperature
 \end{enumerate}

 \end{itemize}

\end{frame}
%-------------------------------------------------------------
%---------------------------------------------------------------------------%
\section{Instructions for PCQ script}
\begin{frame}
\frametitle{Instructions for PCQ script}
The basic procedure of running the PCQ script:
 \begin{itemize}
 \item Generate sample points and quadrature weight
 \item Run simulation with sampled parameters
 \item Parse output file and extract desired properties
 \item Prepare for coefficient computing
 \item compute coefficient by PCQ (or by solving system of equations)
 \item Plot plume height (or mass loading) as a function of random parameters
 \item Plot histogram of plume height (or mass loading) for specific distribution of input parameters
 \end{itemize}
\end{frame}
%-------------------------------------------------------------
%---------------------------------------------------------------------------%
\begin{frame}
\frametitle{Instructions for PCQ script}
\begin{itemize}  
 \item {\bf Generate sample points and quadrature weight}:\\
   \begin{itemize}
    \item The Gaussian Quadrature points generator should be called accordingly based on distribution that you selected:
    Hermite for Gaussian, Legendre for Uniform, Leguerre for Gamma, and Jacobi for Beta (modify smplingx=...)
    \item Transfer and scale sample points (do not need modification)
    \end{itemize}
 \item {\bf Run simulation with sampled parameters}:\\
 Leave it as it is
 \item {\bf Parse output file and extract desired properties}:\\
 Leave it as it is
 \item {\bf Prepare for coefficient computing}:\\
 Leave it as it is
 \item {\bf compute coefficient by PCQ (or by solving system of equations)}:\\
 Leave it as it is    
\end{itemize}
\end{frame}
%-------------------------------------------------------------
%---------------------------------------------------------------------------
 \begin{frame}
 \frametitle{Instructions for PCQ script}
 \begin{itemize}
 \item {\bf Plot plume height (or mass loading) as a function of random parameters}:\\
 Maybe need to modify the xlable and ylabel, title...
 \item {\bf Plot histogram of plume height (or mass loading) for specific distribution of input parameters}:\\
 You can try different number of bins ect. depends on you.
\end{itemize}
\end{frame}
%-------------------------------------------------------------
%---------------------------------------------------------------------------%


% \begin{frame}{References}
%   \begin{thebibliography}{99}
%   \bibitem{one}
% Anis Das Sharma, Alpa Jain, Kong Yu, " Dynamic Relationship and Event Discvery".
% \bibitem{two}
% Nguyen Bach and Sameer Badaskar, Presentation on "Survey on Relation Extraction".
% \bibitem{three}
% Sunita Sarawagi, "Surv"

% \end{thebibliography}
% \end{frame}
% %---------------------------------------------------------------------------%

\begin{frame}
\Large
\begin{center}
 \sc {Thank You \ldots} 
\end{center}
\end{frame}
%---------------------------------------------------------------------------%
\end{document}
